\chapter{State of the Art}
This chapter presents the current state of the art regarding insights extraction on social media.
Many aspects of online users have been explored in order to profile customers.
"Then, there is a focus on what has been done in terms of providing actionable personality insights."

%II
Some studies aimed to identify clear demographic characteristics based on both the analysis of a user's activities and her network inside the social media. 
Twitter is commonly used for the extraction of gender \cite{miller2012gender}, age or age groups \cite{culotta2015predicting}.
Also, a person's family status is inferred through the detection of life events such as the birth of a child and a marriage \cite{dickinson2015identifying}.


%III
The literature also presents many examples of latent attributes extraction.
Some of the most remarkable research has been carried out by the \emph{World Well-Being Project} \footnote{https://wwbp.org/}; a research center which used 
social media to measure attitudes and personal characteristics such as optimism and pessimism \cite{ruan2016finding}, temporal orientation \cite{schwartz2015extracting}.
Many different social networks have been explored as well as many aspects that are not limited only to text but also include images and social interactions.
Finally, it is a common practice inferring behaviour through a variety of personality models.

%IV
However, what has been done is almost completely focused only on the feasibility of extracting attitudes' insights from online activities rather than a commercial use of the obtained information to generate a marketing advantage.
So, the literature presents only a few systems which satisfy the right requirements for an application in the real world, such as those imposed by the GDPR TODO CITE.

\section{Customers Profiling}
A precise and detailed description of social media users requires the analysis of many aspects of social media. 
Indeed, understanding the users means being able to quantify and qualify how they present themselves \cite{schwartz2013personality}.

Many of the systems proposed for social media analysis use as fundamental component features that describe interactions of users, such as the number of followers, mentions, likes, and comments.
%Questa analisi di interaction è stata esplorata molto, partendo da solo numeri statici, all'osservazione dinamica della rete...
This type of analyses has been largely explored since studies about user influence and social engagement. First, raw measures publicly available on social media were used to calculate metrics to represent effectively the user's influence \cite{oviedo2014metric}.
Further research proved that simply observing ground numbers of a profile can lead to a misunderstanding. Cha stated that the indegree alone (number of followers) reveals little and suggested to consider shares and mentions from other users \cite{cha2010measuring}.
D. Romero et al. observed influence analysing the propagation of web links over time using both the structural properties of the network as well as the diffusion behaviour among users \cite{romero2011influence}. They also regarded the \emph{passivity of a user}, a measure of how difficult it is for other users to influence him, and used it to weigh the tweets propagation network. 
%Cambia da social a social a seconda di unilaterale o bilaterale
Many different networks can be explored on social media in order to identify influence, communities, and trend topics applying the myriad of network concepts and analyses such as degree centrality and modularity \cite{chae2015insights}. 
The nature of these graphs can change regarding the platform's characteristics and the aspect we are looking at. Li complained about undirected networks, such as the Facebook friends graph and proposed a method based on the \emph{Share/Reply/Mention} directed network to capture user influence \cite{li2014social}.
These observations are usually used to profile a person's social environment and to assess his or her role inside it. 

%Dopo di che, si ha iniziato a fare analisi del contenuto
%Prima con hashtag e url, poi con altri servizi
A second fundamental point carried out by literature on social media is the analysis of the context the user is talking about.
Obviously, being aware of what topics drive someone's interactions is essential to profile his or her interest. 
Moreover, they can be used to reduce other types of analyses to a specific field of interest. For example, focussing on users' influence in sports discussions.
To understand context, it is necessary to observe the content of the messages which is usually composed by text and images or videos.
Firstly, keywords in the activities were used to identify topics \cite{cha2010measuring}. This methodology shows some clear issues, especially when used for social media when messages tend to be extremely abbreviated through acronyms and slang words.
Other approaches, feasible in a limited number of platforms, proposed to use most used hashtags to obtain linguistic content starting from the activities \cite{pennacchiotti2011machine}.
Finally, a more general technique is using the tree of Wikipedia categories to characterize the user's interests. 
This method fits well with both text and multimedia content thanks to a number of services that apply semantic analysis techniques to extract relevant entities \cite{torrero2018wikipedia}. 

\section{Behavioural insights}
"Psychometric profiling is the process by which your actions are used to infer your personality."

The literature presents many different techniques for the extraction of behavioural information which are all based on the most used personality models to study specific traits of an individual.
The models proposed are classifiers or regression one depending on which personality taxonomy is being applied.
Each model is specialized to a single specific personal characteristic, for example, using the MBTI personality model there are 4 classifiers, one for each cognitive function.
Almost all models presented work on social user composed by the totality, or a portion, of their timeline rather than single activities since linguistic information contained by a single short activity is not enough to accurately predict personality aspects \cite{mairesse2007using}.

The feature extraction shares some fundamental aspects in the majority of systems. The results of the analysis seen before represent two essential groups.
Indeed, understanding a user's network helps understand how he or she reacts to external stimuli. Therefore, it plays a crucial role in the extraction of behavioural insights from online activities.
Also, research has shown a strong correlation between discussed topics and personality aspects of a person \cite{kern2016gaining}.
Guntuku et al. proved that studying semantic concepts contained in posted images can give a significant performance gain in predicting personality traits with respect to the \emph{Big Five model} \cite{guntuku2017studying}.
However, the literature contains a very few number of proposals that considered the content of the activity and are usually confined to hashtags and key words in the text \cite{ruan2016finding}. 

Regarding features that describe the social presence of a person. These are usually included by the majority of models. Although some are limited to basic information such as the number of followers, following or friends, the number of activities, and their frequency \cite{quercia2011our}.
Over time, the literature presented the application of more complex features, obtained as results from further analysis of the user's network such as interaction patterns by a person towards the author of the post \cite{dickinson2015identifying}.
For example, significative patterns could be a high retweet ratio by users who do not retweet much other sources by or an elevate number of interactions by users with many followers. However, these last observations need the permission of each person belonging to the analysed network to be respectful of GDPR requirements. Thus, even though they could give great results, their lawful application in the market is quite intricate.

Then, there is a third fundamental group of features which is probably the most important one.
Since psychological studies proved that there is an effective relationship between linguistic style and personality aspects, understanding detailly how an individual writes is a crucial step \cite{pennebaker1999linguistic}.  
Some of the most common and basic features are word counts, sentences per activity, word per sentence, and punctuation count. These have been applied by the majority of models with great results in many different environments.
For example, Farnadi recognized personality of YouTube vloggers using the script of their videos to extract this linguistic information \cite{farnadi2014multivariate}.
Furthermore, more recent studies have tested features from specialized and complex tools for text analysis. These can reveal precisely thoughts, feelings, and motivations of the text's author.
The \emph{Linguistic Inquire and Word Count (LIWC)} developed by Tausczik and Pennebaker is certainly the most used one \cite{tausczik2010psychological}.
Other services that have been tested are the \emph{MRC Psycholinguistic Database} and the \emph{NLPRO}, developed by oNLPLAB \cite{wilson1988mrc, tsarfaty2018natural}. Lima et al. tested the three of them concluding with the first one as the most performing one \cite{lima2019tecla}

Far vedere diversi tipi di analisi fatti per la personalità
Poi dire che molti si basano sul big5 e far vedere quelli sull'MBTI
\section{Commercial applications}