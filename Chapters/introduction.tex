\chapter{Introduction}
During my internship at U-Hopper, I had the opportunity to develop this Thesis as a result of my experience inside the company. 
\textbf{ U-Hopper is a research-intensive deep-tech SME, headquartered in Trento, providing big data-enabled solutions and technologies for the government, retail and manufacturing sectors. U-Hopper has received numerous awards for its innovative solutions, including, among the others, the Lamarck prize (2013), a EC Seal of Excellence (2015), the Innov@Retail prize (2016) and a nomination for the 2017 EC Innovation Radar Awards.}.
The company is active in many different domains such as retail and tourism and offers a variety of competences including chatbots, analytics, and machine learning.
Thanks to Tapoi\footnote{\url{www.tapoi.me}}, an innovative data intelligence solution, U-Hopper is also into the sector of user profiling.
It allows businesses to deliver personalized experiences to their customers through the mining and analysis of their activities on social networks.
Thus, the extraction of behavioural insights can be a valuable aspect since being aware of how an individual comes to a decision helps to provide each customer with the right tailored content.

According to neuroscientists Adelstein et al., personality describes human behavioural responses to wide classes of external stimuli. It works as an adaptive system \cite{adelstein2011personality}.
Psychologists claim that in this system, the constructs represent interacting psychological mechanisms that cause the flow of behaviour and experience; and the adaptive function allows the whole organism to survive and fulfil its needs \cite{deyoung2010toward}.
The parameters of the adaptive system represent the variation of the same from person to person and, therefore, characterize uniquely every individual. These parameters are also referred to as personality traits in several different personality models studied over the years.
Each model includes its range of traits which combinations describe several personality types.

Psychology researchers have shown clear connections between general personality traits and many types of behaviour.
First of all, some fundamental traits describe the type of relationship a person has with the outside world and the way he or she communicates \cite{lima2016predicting}.
In the field of Human-Computer Interaction, users prefer interfaces designed to represent personalities that most closely matched their own \cite{nass2000does}.
Some studies have also suggested connections between customer personality and marketing. Through techniques more focused on the target audience, it is possible to profile individuals, and tailor advertisement automatically displayed based on their personality.\cite{bachrach2012personality}
Therefore, the ability to identify the personalities of people or even better, details of their personality traits is a significant competitive advantage since
we would have a well-defined representation of the customer's reasoning process.

\section{Personality models}
In psychology, an individual's personality is described through a specific model comprised of different traits to provide a better understanding of human's unique characteristics.
\textbf{Qualcosa sull'importanza di un modello per la personalità TODO, poi spostare la riga sotto all'interno della subsection}
While several models exist, the \textit{Big Five}, also known as the \textit{five-factor model} and the \textit{OCEAN model} is one of the most well-researched and widely accepted taxonomies among scientists.\cite{mccrae1992introduction, mccrae1987validation}
\subsection{Big-Five factor structure}
It formalizes personality along 5 domains, namely Openness, Conscientiousness, Extroversion, Agreeableness, and Neuroticism. Each one of these traits is continuous and usually ranges on a scale from 1 to 5.
At the two extremes of each trait, two separate aspects reflect a particular behaviour. 
High openness marks imagination, creativity, and curiosity in learning and exploring new things. Conscientiousness represents self-discipline and attention to details.
Extroversion measures preferences for interacting with other people. Agreeableness reflects the extent to which a person is generous, trustworthy and always willing to help others.
Finally, a high score on neuroticism indicates a tendency to get stuck in negative emotions.
For example, conscientiousness is bounded by carelessly at the lowest end and by organization and efficiency at the greatest one.
Also, neuroticism, which quantifies the tendency to experience negative emotions, is characterized respectively by nervousness and confidence.
"Breve descrizione di ognuno dei 5"

Since its first definition, this model rapidly became one of the standards the psychological community, largely accepted by the most share of scientist.
However, concerning the application of personality insights in work environment and marketing received some critics.
"Critica al modello, vedi excel del problema per il paper". "Trova paper su critica al modello per quanto riguarda l'applicazione a campi esterni alla psicologia" \cite{hough2003use, patton2014career}.

\subsection{Myers-Briggs type indicator}
"Introduzione al MBTI, caratteristiche. Dimensioni e problema di classificazione"
"Citazione al paper di overlap tra i due modelli"

\section{Problem statement}
"To facilitate communication, it is possible to use personality models to gain a better understanding of what drives the person of interest."

"The personality of an individual is defined as the set of responses to external stimuli"
\section{Customer insights}
\section{Objectives}
\section{Outline}

Chapter 2 describes the state of the art. It could change once that the system design is defined. Chapter 3 introduces the design solution. It focuses on used components and algorithms, their logic and their interfaces. Chapter 4 shows how the mentioned components are implemented and integrated. It follows the implementation of the algorithm and the evaluation of a general prototype of the proposed system. Chapter 5 concludes the thesis with some observations and future work proposals.